\documentclass[conference]{IEEEtran}

% graphic declaration
\usepackage{graphicx}
\usepackage{listings}
\usepackage[english]{babel}
\usepackage{float}
\usepackage{color}
\usepackage{caption}
\usepackage{indentfirst}
\usepackage{amsmath}
\usepackage{fancybox}

\definecolor{dkgreen}{rgb}{0,0.6,0}
\definecolor{gray}{rgb}{0.5,0.5,0.5}
\definecolor{mauve}{rgb}{0.58,0,0.82}

\iffalse
\lstset{frame=tb,
  language=Java,
  aboveskip=3mm,
  belowskip=3mm,
  showstringspaces=false,
  columns=flexible,
  basicstyle={\footnotesize\ttfamily},
  numbers=none,
  numberstyle=\tiny\color{gray},
  keywordstyle=\color{blue},
  commentstyle=\color{dkgreen},
  stringstyle=\color{mauve},
  breaklines=true,
  breakatwhitespace=true,
  tabsize=3
}
\fi

\lstset{%
  numberstyle=\scriptsize\color[gray]{0.3},
  numbersep=5pt,
  stepnumber=1,
  numbers=left,
  xleftmargin=15pt,
  belowcaptionskip=\bigskipamount,
  captionpos=b,
  escapeinside={*'}{'*},
  language=Java,
  tabsize=2,
  emphstyle={\bf},
  commentstyle=\it,
  stringstyle=\mdseries\ttfamily,
  showspaces=false,
  keywordstyle=\bfseries,
  morekeywords={then,end},
  columns=flexible,
  basicstyle=\scriptsize\ttfamily,
  showstringspaces=false,
  morecomment=[l]\%,
}

\graphicspath{ {data/} }

\ifCLASSINFOpdf
  % \usepackage[pdftex]{graphicx}
  % declare the path(s) where your graphic files are
  % \graphicspath{{../pdf/}{../jpeg/}}
  % and their extensions so you won't have to specify these with
  % every instance of \includegraphics
  % \DeclareGraphicsExtensions{.pdf,.jpeg,.png}
\else
  % or other class option (dvipsone, dvipdf, if not using dvips). graphicx
  % will default to the driver specified in the system graphics.cfg if no
  % driver is specified.
  % \usepackage[dvips]{graphicx}
  % declare the path(s) where your graphic files are
  % \graphicspath{{../eps/}}
  % and their extensions so you won't have to specify these with
  % every instance of \includegraphics
  % \DeclareGraphicsExtensions{.eps}
\fi

% correct bad hyphenation here
\hyphenation{op-tical net-works semi-conduc-tor}


\begin{document}
%
% paper title
% Titles are generally capitalized except for words such as a, an, and, as,
% at, but, by, for, in, nor, of, on, or, the, to and up, which are usually
% not capitalized unless they are the first or last word of the title.
% Linebreaks \\ can be used within to get better formatting as desired.
% Do not put math or special symbols in the title.
\title{Job2P: Job Scheduling APIs for\\Wi-Fi Peer-to-peer Mobile Network}


% author names and affiliations
% use a multiple column layout for up to three different
% affiliations
\author{\IEEEauthorblockN{Le Dinh Minh}
\IEEEauthorblockA{Utah State University\\
minh.le@aggiemail.usu.edu}
\and
\IEEEauthorblockN{Young-Woo Kwon}
\IEEEauthorblockA{Utah State University\\
young.kwon@usu.edu}}


% make the title area
\maketitle

\begin{abstract}
These days, along with the rapid revolution of mobile device industry, software is also being built heavier and consumes more CPU performance and energy than they were in the past few years. Since the modern mobile devices support multiple connection methods like Wi-Fi, Bluetooth or NFC, the requirements of sharing the workloads and resources among the devices within a network to reduce full workload on an arbitrary device are being considered, especially in the areas Internet is not available, which can be found anywhere.

To address those limitations on mobile devices, we proposed Job2P as the Job Scheduling APIs for Android development that leverages Wi-Fi Direct to support sharing workloads and resources over peer-to-peer mobile device network, which doesn't require any Internet connections. Job2P provides a simple and straightforward API interface to get rid of sophistication of network implementation, letting developers easily create their distributed mobile applications with capability of forming closed range network. In term of workload distribution, Job2P splits task and resource into parts and packs them into the smaller units called jobs and dispatch to the peers. To distributed jobs equitably among the peers, a decision maker is added to decide the amount of resource the peer has to handle bases on its percentage of availability. Moreover, our APIs can handle fault tolerant for network malfunctions. 

Through our case studies, we realized that Job2P improves up to 45\% performance of running task in term of performance and energy consumption. 
\end{abstract}


% For peer review papers, you can put extra information on the cover
% page as needed:
% \ifCLASSOPTIONpeerreview
% \begin{center} \bfseries EDICS Category: 3-BBND \end{center}
% \fi
%
% For peerreview papers, this IEEEtran command inserts a page break and
% creates the second title. It will be ignored for other modes.
\IEEEpeerreviewmaketitle



\section{Introduction}
These days, mobile phones have been playing a very important role in human society. People need their cellphone everyday for surfing Internet, searching for information, online shopping, communicating with friends, taking and sharing photos etc. As a result of highly intensive completions between the global mobile phone manufactures like Samsung and Apple, the cellphones are more and more equipped themselves with high specifications (CPU, RAM and built-in storage) and top functionality so that they are powerful enough in compare with those in the past or even with a desktop. 

With the hardware rapid development, software is also being built more and more sophisticated in many aspects. They may occupy more space on local storage, use more memory, and thus consume more energy. Even though the software is designed to run on mobile device, its size may vary up to 1GB (like image processing or image recognizing) and memory consumption can be up to 1GB as well. Since there are possibly many heavy workloads running on a device at some points, the needs of sharing workloads among the mobile devices is really necessary with respect to performance and energy efficiency. 

Distributed computing are so far well-developed in PC or embedded networks, bringing the synthesized power of computation from multiple computers to solve a problem. Up to now, there are thousands of solutions for the developers to create a distributed computing system. For instance, the emergency of a number of message-oriented middleware like DataTurbine \cite{rbnb}, RabbitMQ \cite{rabbitmq} or NaradaBrokering \cite{naradabrokering} lessened the effort of development for distributed applications to the minimum but still maintained all equipped functionality like data mirroring, dynamic network topology or fault tolerant. However, despite of stunning equipments, it is revealed that the most disadvantage of this kind of applications is the dependence of network connection or wireless access point. Without a network established, nothing happens.

Unlike the other PCs or wired devices, the mobile devices have their own advantage of multiple non-equivalent network capability. One of the remarkable network capability is installed on modern mobile devices are Wi-Fi Direct, which allows them to discover the others in any short distance less than 200 meters without utilizing Internet and wireless access point. No internet connections are required, and it will help the owner to connect to the devices within a closed distance. By establishing connection between the two devices to form a pair, Wi-Fi Direct can provide the simple way to dynamically initiate a peer-to-peer network. Available on Android devices from version 4.0 (which more than 96\% of devices are using these days), as well as a number of Intel-featured laptops and game consoles, there is the high possibility of discovering the other mobile devices at anywhere.

To address those limitations with heavy software on mobile devices, we proposed Job2P as the Job Scheduling APIs for Android development that leverages Wi-Fi Direct to support sharing workloads and resources over peer-to-peer mobile device network, which doesn't require any Internet connections. Job2P provides a simple and straightforward API interface to get rid of sophistication of network implementation, letting developers easily create their distributed mobile applications with capability of forming closed range network. In term of workload distribution, Job2P splits task and resource into the smaller units called jobs, and dispatch to the peers. To distributed jobs equitably among the peers, a decision making module is added to decide the amount of resource the peer has to handle bases on its percentage of availability. Moreover, our APIs can handle fault tolerant for network malfunctions.

Based on our experiments and results, our contributions are as follow
\begin{itemize}
	\item \textbf{Simple Job Scheduling APIs} a library for developers facilitates establishing mobile peer-to-peer network.
	\item \textbf{Job Scheduler} for job manipulation and resource heterogeneous adaptation
	\item \textbf{Decision making module} determines a list of appropriate devices to execute bases on applied algorithm on availability
	\item \textbf{High productivity results} our test case experiments will give the proof of APIs productivity
\end{itemize}
 


% An example of a floating figure using the graphicx package.
% Note that \label must occur AFTER (or within) \caption.
% For figures, \caption should occur after the \includegraphics.
% Note that IEEEtran v1.7 and later has special internal code that
% is designed to preserve the operation of \label within \caption
% even when the captionsoff option is in effect. However, because
% of issues like this, it may be the safest practice to put all your
% \label just after \caption rather than within \caption{}.
%
% Reminder: the "draftcls" or "draftclsnofoot", not "draft", class
% option should be used if it is desired that the figures are to be
% displayed while in draft mode.
%
%\begin{figure}[!t]
%\centering
%\includegraphics[width=2.5in]{myfigure}
% where an .eps filename suffix will be assumed under latex, 
% and a .pdf suffix will be assumed for pdflatex; or what has been declared
% via \DeclareGraphicsExtensions.
%\caption{Simulation results for the network.}
%\label{fig_sim}
%\end{figure}

% Note that the IEEE typically puts floats only at the top, even when this
% results in a large percentage of a column being occupied by floats.


% An example of a double column floating figure using two subfigures.
% (The subfig.sty package must be loaded for this to work.)
% The subfigure \label commands are set within each subfloat command,
% and the \label for the overall figure must come after \caption.
% \hfil is used as a separator to get equal spacing.
% Watch out that the combined width of all the subfigures on a 
% line do not exceed the text width or a line break will occur.
%
%\begin{figure*}[!t]
%\centering
%\subfloat[Case I]{\includegraphics[width=2.5in]{box}%
%\label{fig_first_case}}
%\hfil
%\subfloat[Case II]{\includegraphics[width=2.5in]{box}%
%\label{fig_second_case}}
%\caption{Simulation results for the network.}
%\label{fig_sim}
%\end{figure*}
%
% Note that often IEEE papers with subfigures do not employ subfigure
% captions (using the optional argument to \subfloat[]), but instead will
% reference/describe all of them (a), (b), etc., within the main caption.
% Be aware that for subfig.sty to generate the (a), (b), etc., subfigure
% labels, the optional argument to \subfloat must be present. If a
% subcaption is not desired, just leave its contents blank,
% e.g., \subfloat[].


% An example of a floating table. Note that, for IEEE style tables, the
% \caption command should come BEFORE the table and, given that table
% captions serve much like titles, are usually capitalized except for words
% such as a, an, and, as, at, but, by, for, in, nor, of, on, or, the, to
% and up, which are usually not capitalized unless they are the first or
% last word of the caption. Table text will default to \footnotesize as
% the IEEE normally uses this smaller font for tables.
% The \label must come after \caption as always.
%
%\begin{table}[!t]
%% increase table row spacing, adjust to taste
%\renewcommand{\arraystretch}{1.3}
% if using array.sty, it might be a good idea to tweak the value of
% \extrarowheight as needed to properly center the text within the cells
%\caption{An Example of a Table}
%\label{table_example}
%\centering
%% Some packages, such as MDW tools, offer better commands for making tables
%% than the plain LaTeX2e tabular which is used here.
%\begin{tabular}{|c||c|}
%\hline
%One & Two\\
%\hline
%Three & Four\\
%\hline
%\end{tabular}
%\end{table}


% Note that the IEEE does not put floats in the very first column
% - or typically anywhere on the first page for that matter. Also,
% in-text middle ("here") positioning is typically not used, but it
% is allowed and encouraged for Computer Society conferences (but
% not Computer Society journals). Most IEEE journals/conferences use
% top floats exclusively. 
% Note that, LaTeX2e, unlike IEEE journals/conferences, places
% footnotes above bottom floats. This can be corrected via the
% \fnbelowfloat command of the stfloats package.


\section{Related Works}

\section{Approach}
To retrieve the goals, as well as making our APIs widely adaptive to different context and usages, our penetrated idea of design is hiding the complexity of system implementation and open to developers the capability of customization. This below figure describes the internal architecture of a typical application utilizing our APIs to form a distributed mobile peer-to-peer system (Figure \ref{fig:architecture}). Basically, our system comprises of two main components:

\begin{figure}[H]
\centerline {
\includegraphics[width=0.39\textwidth, natwidth=643, natheight=559]{data/jobShareArch}
}
\caption{Architecture of a typical application implementing Job2P}
\label{fig:architecture}
\end{figure}

\begin{itemize}
	\item \textbf{Job Handler} splits the task into jobs, each contains a process definition and resources, and base on making decision on conditions of peers, dispatch jobs to the appropriate peers. Moreover, it handles merging results completed by peers by predefined splitting and merging mechanisms.
	\item \textbf{Wi-Fi Peer-to-peer Broadcaster} hides the details implementation and provide a simpler API interface for forming peer-to-peer network.
\end{itemize}

\subsection{APIs for Peer-to-peer Network}
To easily form up multiple pair connections between devices in the network, we utilized Wi-Fi Direct, the new feature available on Android 4.0 and later. \texttt{Wi-Fi Peer-to-peer Broadcaster} (WPB) module wrapped up the complexity of Wi-Fi Direct library of Android APIs and simplify the functions to minimum to let developers get rid of low level network implementation. These below steps figures out a way to simply initiate a mobile network.

When the app starts up, WPB will call \texttt{discoverPeers()} to send a message to the other peers to let them know its availability. Once a peer receives such message WPB will update the list of devices, therefore reform the network. See figure \ref{fig:forming}

\begin{figure}[H]
\centerline {
\includegraphics[width=0.5\textwidth, natwidth=1127, natheight=615]{data/discoverPeers}
}
\caption{Forming peer-to-peer network}
\label{fig:forming}
\end{figure}

When peers have information of the others, they are ready to connect. We provide function \texttt{connectToADevice()} holding \texttt{WifiP2pDevice} as one input parameter to invoke the handshake with another device. Once connection established, the device actively invoked will be assigned as client, which utilize the \texttt{ClientSocketHandler} to listen to a client socket. The device passively received the invocation will be promoted as server and use the \texttt{ServerSocketHandler} to initiate server socket with an random port. If the invoked device is already a server, it will maintain its current state and continuously accept the new client. Moreover, since a device can either be a server to some peers, or serve as client to the others, this method is able to build up a close range peer-to-peer network.

\subsection{Job definition} \label{job-description}
We give the job definition to developers. To avoid any misconfiguration from remote execution, we provide a common Job definition interface

\noindent \shadowbox{%
\begin{minipage}{1\columnwidth}
	\captionof{lstlisting}{Job Definition}
	
	\begin{lstlisting}
	public class Job {
		public Object exec(Object originalObject) {
			// job implementation
		}
	}
	\end{lstlisting}
	
\end{minipage}}	

Since Dalvik VM didn't support loading class the way Java VM does. Instead it will load Dalvik execution ("dex") files from an alternative locations such as internal storage or network. We provide \texttt{DexCreator} tool, the windows application supports compiling the \texttt{Job} java file into the dex job package (a jar file). Dex package and splitted resources in binary format will be added into one \texttt{JobData} object by \texttt{JobDispatcher}, signed with checksum for consistency and dispatched to the other peers.

When client peer receives an object of \texttt{JobData} sent from server, it will firstly check checksum to confirm the consistency, then deserialize it into \texttt{Job} and resources and execute the job with that data.

\subsection{Job Scheduling and Decision Maker} \label{scheduling}

In peer-to-peer network supported by our APIs, the server will distribute jobs to peers bases on investigating their availability by \texttt{DecisionMaker} (Figure \ref{fig:checkStatus}). At the early stage before the transmission, decision maker sends request for status to some of selected peers in its network. If one peer receives this request, it will calculate its capability of response using the measurements CPU, memory and battery usage at a certain point of time. 

Retrieved from \texttt{/proc/stat} system file \cite{stat_explain}, the percentage of CPU usage is expressed by this following expression in two short consecutive times

$$Usage_{CPU} = \frac{(\sum{T_{CPU2}} - T_{Idle2}) - (\sum{T_{CPU1}} - T_{Idle1})}{(\sum{T_{CPU2}} - \sum{T_{CPU1}})}$$

When $\sum{T_{CPU}}$ is total time of running CPU and $T_{Idle}$ is idle time correspondingly in hertz. In term of memory usage, the $Usage_{Mem}$ can be determined by using \texttt{MemoryInfo} from Android API to retrieve $Mem_{Avail}$ and $Mem_{Total}$, so the $Usage_{Battery}$.

\begin{figure}[H]
\centerline {
\includegraphics[width=0.45\textwidth, natwidth=915, natheight=837]{data/checkStatusFlow}
}
\caption{Decision making workflow}
\label{fig:checkStatus}
\end{figure}

Since in the mobile device, the lower resource usage state states the higher availability, and the higher specifications represents the better responsibility. Then the level of responsibility of device can be simply summarized by the below expression

$$RL = \frac{CPU_{Spec}}{Usage_{CPU}} + \frac{Mem_{Spec}}{Usage_{Mem}} + \frac{Battery_{Spec}}{Usage_{Battery \times 1000}}$$

Where $CPU_{Spec}$ has GHz unit, $Mem_{Spec}$ has GB unit and $Battery_{Spec}$ has uAh unit. Especially $CPU_{Spec}$ is determined by number of its core. For example, a quad-core CPU at speed of 1.3GHz can be counted as $1.3 \times 4$ or 5.2. At a certain time of that device, if $Usage_{CPU}$ is 0.3, $Usage_{Mem}$ is 0.5 (half of 1GB memory consumed), $Usage_{Battery}$ is 0.7 or 70\% used over a 2800uAh capacity battery, the value of $RL$ will be

$$RL = \frac{5.2}{0.3} + \frac{1}{0.5} + \frac{2800}{0.7 \times 1000} = 23.33$$

In a peer-to-peer network comprising of $n$ devices, where $i$-device has responsibility level $RL_{i}$, the \texttt{DecisionMaker} will assign the job with carrying amount of data ($M_{i}$) which is equivalent to

$$M_{i} = M\frac{RL_{i}}{\sum_{j = \overline{1,n}}{RL_{j}}}$$

Where $M$ is total size of data in bytes.

\section{Experiments and Results}

To measure the performance of the system equipped with our APIs, we decorated a small testbed with collaboration of 5 different Android devices to perform our 3 test cases:
\begin{itemize}
	\item \textbf{Image Processing} we will initiate the peer-to-peer network test to perform blurring a large scale image which is unable to process at any of our devices. Particularly, to process an image with size $4000 \times 4000$ and 4 bytes to express each pixel color, application must spare the amount of memory equivalent to 64MB which is too expensive for the system, occasionally this kind of image would be refused to load. 
	\item \textbf{Text analysis} another proof of performance on text processing. Our APIs will yield the data parser to developer for data heterogeneous adaptation.
	\item \textbf{GPS} establishing a GPS connection is proof of high energy consuming. It would be impossible for a device with low battery to keep update with GPS frequently. We will base on Job2P to build a simple system so that one device can benefit GPS locations from the healthier devices.
\end{itemize}

\subsection{Preliminary Estimations}
Assume that we have a Wi-Fi peer-to-peer network available with $n$ devices, each device at a certain time has level of responsibility $RL_{i} (i = \overline{1,n})$. According to the section \ref{scheduling}, if $E$ is the energy consumed by the application for only completing the task regardless of other ambiances, the total energy $E_{0}$ will be

$$E_{0} = E + E_{w}$$

Where $E_{w}$ is energy the app requires for waiting. Also, 

$$E_{p2p} = E(\frac{RL_{0}}{\sum_{i = 1}^{n}{RL_{i}}}) + E_{WiFi} + E_{w}$$ 

Where $E_{WiFi}$ is energy consumed by Wi-Fi for sending jobs to other peers.

In this estimation we skipped considering $E_{w}$ since it will depend on appearance of applications. If application have no GUI like system background services, $E_{w}$ will cause very little effect. From the two above equations, we can get the difference energy consumption between the two job processing mechanisms $E_{Diff}$ 

$$E_{Diff} = E_{0} - E_{p2p}$$ or $$E_{Diff} = E(1 - \frac{RL_{0}}{\sum_{i=1}^{n}{RL_{i}}}) - E_{WiFi}$$

According to \cite{wifi_energy}, Wi-Fi caused battery drained linearly by time during the transmission, particularly the drain can be represented by $y = 17.01x - 0.93$ for downloading and $y = 17.31x - 2.28$ for uploading. Therefore, it is inferred that in a mobile system with a certain number of devices in different levels of responsibility, if E is big enough, in other words, if the task to perform is big enough, then $E_{Diff} > 0$ will happen, thus deploying a peer-to-peer cluster will give the great benefit in term of energy efficiency. The bigger value of $E_{Diff}$, the more benefit we will archive.

\subsection{Results}

\subsubsection{Applying Job2P to an Android project}
The library should be simple, so that developer can integrate within a few steps.

First of all, developer needs to define the \texttt{UIHandler} to receive messages from system when it goes in real-time. While system is in progress, it will periodically return \texttt{MAIN\_INFO} to inform logs or statuses. When job-results have been collected from all the peers, the \texttt{MAIN\_JOB\_DONE} message will be thrown along with the final proceeded and merged data.\\


\noindent \shadowbox{%
\begin{minipage}{1\columnwidth}
	\captionof{lstlisting}{UI handler}

	\begin{lstlisting}
	Handler mainUiHandler = new Handler() {
		@Override
		public void handleMessage(Message msg) {
			switch (msg.what) {
				case Utils.MAIN_JOB_DONE: {
					// when job is completely finished
				}
				case Utils.MAIN_INFO: {
					// to receive messages from the processor
				}
			}
		}
	};
	\end{lstlisting}

\end{minipage}}

Secondly, developer needs to declare \texttt{DataParser} to determine data-type and parser to equip for manipulating data at run-time (see sub section \ref{job-description}). \texttt{JobHandler} is the main component which wraps up the complexity , and exposes only the necessary functions like \texttt{discoverPeers()} and \texttt{dispatchJob()}. To send ACK messages to other peers for exchanging acknowledgments and reforming network, we need to call \texttt{discoverPeers()} function on the program, this work should be done as soon as application starts. 

When network is formed and connections are held from some of the peers, \texttt{dispatchJob()} will be call to locate the resources and job which predefined in local storage, it then invokes \texttt{DecisionMaker} (sub section \ref{scheduling}) for job splitting and binary serialization. Finally jobs will be dispatched over the socket\\

\noindent \shadowbox{%
\begin{minipage}{1\columnwidth}
	\captionof{lstlisting}{Declare DataParser and JobHandler}

	\begin{lstlisting}
	// data parser: to determine datatype how to split the data 
	dataParser = new BitmapJobDataParser();
	...
	// handlers registration
	jobHandler = new JobHandler(this, dataParser);
	jobHandler.setSocketListener(
		new JobHandler.JobSocketListener() {
			@Override
			public void socketUpdated(... isConnected) {
					// when socket is updated
			}
	});
	...
	// update the device list
	deviceList.setAdapter(
		jobHandler.getDeviceListAdapter());
	...
	// send ACK to other members to reconstruct network
	jobHandler.discoverPeers();
	...
	// address resources and job to execute
	String dataPath = downloadPath + "/mars.jpg";
  String jobPath = downloadPath + "/Job.jar";
  jobHandler.dispatchJob(dataPath, jobPath);

	\end{lstlisting}

\end{minipage}}



\subsubsection{Performances}
Diagrams go here

\section{Conclusions}
The conclusion goes here.




% conference papers do not normally have an appendix


% use section* for acknowledgment
\section*{Acknowledgment}


The authors would like to thank...





% trigger a \newpage just before the given reference
% number - used to balance the columns on the last page
% adjust value as needed - may need to be readjusted if
% the document is modified later
%\IEEEtriggeratref{8}
% The "triggered" command can be changed if desired:
%\IEEEtriggercmd{\enlargethispage{-5in}}

% references section

% can use a bibliography generated by BibTeX as a .bbl file
% BibTeX documentation can be easily obtained at:
% http://mirror.ctan.org/biblio/bibtex/contrib/doc/
% The IEEEtran BibTeX style support page is at:
% http://www.michaelshell.org/tex/ieeetran/bibtex/
%\bibliographystyle{IEEEtran}
% argument is your BibTeX string definitions and bibliography database(s)
%\bibliography{IEEEabrv,../bib/paper}
%
% <OR> manually copy in the resultant .bbl file
% set second argument of \begin to the number of references
% (used to reserve space for the reference number labels box)
\begin{thebibliography}{100}

\bibitem{rbnb}
Sameer Tilak, Paul Hubbard, Matt Miller, and Tony Fountain, \emph{The Ring Buffer Network Bus (RBNB) DataTurbine Streaming Data Middleware for Environmental Observing Systems}, p125-133, e-Science and Grid Computing, Bangalore 2007.

\bibitem{rabbitmq}
Maciej Rostanski, Krzysztof Grochla, Aleksander Seman, \emph{Evaluation of highly available and fault-tolerant
middleware clustered architectures using RabbitMQ}, p879-884, FedCSIS 2014.

\bibitem{naradabrokering}
Gadgil, H.; Fox, G.; Pallickara, S.; Pierce, M. \emph{Managing grid messaging middleware}, Challenges of Large Applications in Distributed Environments, p83-91, 2006 IEEE

\bibitem{classloader}
Fred Chung, \emph{Custom Class Loading in Dalvik}, http://android-developers.blogspot.com/2011/07/custom-class-loading-in-dalvik.html, July 2011

\bibitem{stat_explain}
\emph{/proc/stat Explained}, http://www.linuxhowtos.org/System/procstat.htm

\bibitem{wifi_energy}
Kalic, G., Bojic, I., Kusek, M., \emph{Energy consumption in android phones when using wireless communication technologies}, p754-759, MIPRO May 2012

\bibitem{live_video_collaboration}
Marco de Sa, David A. Shamma, Elizabeth F. Churchill, \emph{Live mobile collaboration for video production: design, guidelines, and requirements}, p693-707, Journal of Personal and Ubiquitous Computing, Volume 18 Issue 3, March 2014

\bibitem{coast}
Cong Shi, Kaustubh Joshi, Rajesh K. Panta, Mostafa H. Ammar, Ellen W. Zegura, \emph{CoAST: collaborative application-aware scheduling of last-mile cellular traffic}, p245-258, MobiSys 2014

\bibitem{rio}
Ardalan Amiri Sani, Kevin Boos, Min Hong Yun, and Lin Zhong, \emph{Rio: a system solution for sharing i/o between mobile systems}, p259-272, MobiSys 2014


\end{thebibliography}

\end{document}


